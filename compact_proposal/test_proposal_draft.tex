\documentclass[draft]{compact_proposal}

\usepackage[utf8]{inputenc}
\usepackage[english]{babel}

\usepackage{blindtext}
% \usepackage{showframe} % uncomment for frames of boxes in the document.

\title{Example of a Compact Proposal}
\author{Eric Barefoot}
\date{\today}

\begin{document}

\maketitle

\section{This is a section}
Is there a difference between this text and some nonsense like ``Huardest gefburn''? Kjift – not at all!
A blind text like this gives you information about the \mnote{margin note can be used without highlighting} selected font, how the letters are written. A cnote with no arguments is at the end of this sentence\cnote.
A cnote with optional text is here:\cnote[this is optional cnote text] done!
This text should contain all letters of the alphabet and it should be written \mnote[This is optionally highlighted mnote text]{This is a margin note.} the original language.
Pnote is used in this sentence here with no arguments: \pnote and that was it! ...but the length of words \pnote[this is a personal inline note] should match the language.



\subsection{testing mnotes}

This block of text just tests mnotes in a variety of contexts.

This sentence has an mnote with no spaces:\mnote[highlighting this text]{and margin note this text.}.

This sentence has an mnote with space before: \mnote[highlighting this text]{and margin note this text.}.

This sentence has an mnote with space before and after: \mnote[highlighting this text]{and margin note this text.} .

This sentence has an mnote with space after:\mnote[highlighting this text]{and margin note this text.} .

\mnote[This sentence starts]{mnote!} with an mnote highlight.

This sentence has an mnote without any highlighting, that starts right here \mnote{mnote!}, and ends before the comma. 

This sentence has an mnote without any highlighting, that starts right here \mnote{mnote!} but has no comma. 



\subsection{testing pnotes}

This block of text just tests pnotes in a variety of contexts.

This sentence has an pnote with no spaces:\pnote[highlighting this text].

This sentence has an pnote with space before: \pnote[highlighting this text].

This sentence has an pnote with space before and after: \pnote[highlighting this text] .

This sentence has an pnote with space after:\pnote[highlighting this text] .

\pnote[stupid starting...] This sentence starts with an pnote.

\pnote This sentence starts with a pnote with no argument.



\subsection{testing cnotes}

This block of text just tests cnotes in a variety of contexts.

This sentence has an cnote with no spaces:\cnote[highlighting this text].

This sentence has an cnote with space before: \cnote[highlighting this text].

This sentence has an cnote with space before and after: \cnote[highlighting this text] .

This sentence has an cnote with space after:\cnote[highlighting this text] .

\cnote[stupid starting...] This sentence starts with an cnote.

This sentence demonstrates natural use of the cnote \cnote.

\cnote This sentence starts with an cnote with no argument.



\subsection{A subsection this is}
\blindtext[1]
\begin{tightitemize}
	\item one1
	\item two
	\item three
\end{tightitemize}



\section{This is another section}
\blindtext[6]

\end{document}
