\documentclass[draft]{compact_proposal}

\usepackage[utf8]{inputenc}
\usepackage[english]{babel}
 
\usepackage{blindtext}
\usepackage{showframe}

\title{Example of a Compact Proposal}
\author{Eric Barefoot}
\date{February 2018}

\begin{document}
 
\maketitle


\section{This is a section}
\blindtext[4]


\section{This is another section}
Is there a difference between this text and some nonsense like ``Huardest gefburn''? Kjift – not at all!  \cnote{this needs a ref}
A blind text like this gives you information about the selected font, how the letters are written and \pnote{wtf?} an impression of the look.
This text should contain all letters of the alphabet and it should be written \mnote{in of}{typo! check this stuff!} the original language. 
There is no need for special content, but the length of words should match the language. 

\subsection{A subsection this is}
\blindtext[1]
\begin{tightitemize}
	\item one
	\item two
	\item three
\end{tightitemize}
\blindtext[4]

\end{document}